\chapter{Εισαγωγή}
\lhead{Κεφάλαιο 1. \emph{Εισαγωγή}}

\begin{section}{Context-aware εφαρμογές}
\bigskip
Ένα σύστημα θεωρείται context-aware αν χρησιμοποιεί τα συμφραζόμενα (context) εκτέλεσης για να παράσχει συσχετισμένη πληροφορία ή υπηρεσία στο χρήστη, όπου η σχετικότητα εξαρτάται από το στόχο του χρήστη. Τα συμφραζόμενα χωρίζονται σε τέσσερις ευρύτερες κατηγορίες:
\begin{itemize}
\item \emph{Υπολογιστικά συμφραζομενα} όπως συνδεσιμότητα σε δίκτυο, bandwidth επικοινωνίας, πόροι συστηματος. 
\item \emph{Συμφραζόμενα χρήστη} προφίλ του χρήστη, γεωγραφική θέση, δραστηριότητα, συναισθηματική κατάσταση.
\item \emph{Φυσικά συμφραζόμενα} φωτισμός, θόρυβος περιβάλλοντος, θερμοκρασία.
\item \emph{Χρονικό πλαίσιο} ώρα της ημέρας, εποχη του χρόνου. \cite{ctxt_aware_comp}
\end{itemize}
Οι σύγχρονες κινητές συσκευές, παρέχουν του υλικό και λογισμικό για να αντλήσουν πληροφορίες για τα συμφραζομενα και να υποστηρίξουν συσχέτιση περιεχομένου σε πραγματικό χρόνο. Αυτές οι συσκευές ενσωματώνουν τεχνολογίες ικανές να ανιχνεύσουν την κατάσταση, το περιβάλλον και την δραστηριότητα του χρήστη. Τεχνολογίες όπως το GPS, γυροσκόπιο, επιταχυνσιόμετρο, πυξίδα, αισθητήρες εγγύτητας, αισθητήρες φωτός. Το λογισμικό εκμεταλλευόμενο αυτές τις τεχνολογίες μπορεί να παρέχει στοχευμένες υπηρεσίες και να ενισχύσει την εμπειρία χρήσης.
\par
Η γεωγραφική θέση του χρήστη διαδραματίζει πρωταγωνιστικό ρόλο στην εξαγωγή συμφραζομένων μέσω της αναγνώρισης του περιβάλλοντος χρήσης λογισμικού. Με επίκεντρο τη θέση του χρήστη έχουν αναπτυχθεί υπηρεσίες αναζήτησης και απεικόνισης συσχετισμένου περιεχομένου. Η προβολή σημείων ενδιαφέροντος (Points of Interest ή POI) περιλαμβάνει στοιχεία του χάρτη που καταλαβάνουν σημειακές θέσεις σε αντίθεση με γραμμικά δεδομενα όπως δρόμοι και σύνοριακές γραμμές. Τέτοια σημεία χρησιμοποιούνται κατά κόρον σε εφαρμογές πλοήγησης. Τα σημεία ενδιαφέροντος λαμβάνουν διαφορετική σημασία, ανάλογα με το πεδίο εφαρμογής. Τέτοια γεωγραφικά δεδομένα μπορεί να περιλαμβάνουν δημόσια κτήρια, σταθμούς μέσων μαζικής μεταφοράς, σχολεία, νοσοκομεία, αρχαιολογικούς χώρους και πολλά άλλα.

\end{section}


\begin{section}{Στόχος της εργασίας}
\bigskip
Στόχος της εργασίας είναι να ερευνήσει και να υλοποιήσει ένα σύστημα που εκμεταλλεύεται τις σύγχρονες mobile τεχνολογίες εντοπισμού και συγκειμένου για να παρουσιάσει στο χρήστη συσχετισμένο περιεχόμενο. Για το σκοπό αυτό υλοποιήθηκε εφαρμογή για το λειτουργικό κινητών συσκευών Android. Σαν use case επιλέχθηκε η εύρεση πρατηριών καυσίμων. Η εφαρμογή δειγματοληπτεί τη γεωγραφική θέση του χρήστη για να τροφοδοτήσει με τα κοντινότερα πρατήρια. Παράλληλα υλοποιήθηκε υπηρεσία αποθήκευσης και παροχής δεδομένων μέσω του διαδικτύου βασισμένη στο Elasticsearch και το Spring framework.

\end{section}


\begin{section}{Διάρθρωση της εργασίας}
\bigskip
Αρχικά γίνεται μια εισαγωγή στο Android περιβάλλον. Στόχος αυτού του κεφαλαίου είναι να παρουσιάσει στον αναγνώστη, εν συντομία, το λειτουργικό, το περιβάλλον ανάπτυξης εφαρμογών και τις τεχνολογίες του API που χρησιμοποιήθηκαν. Περιγράφονται συνοπτικά τα επίπεδα του Android stack και γίνεται μια εισαγωγή στα βασικά μέρη μιας εφαρμογής Android. Τέλος αναλύονται οι υπηρεσίας τοποθεσίας.
\par
Στο τρίτο κεφάλαιο γίνεται μια συνοπτική παρουσίαση της εφαρμογής. Περιγράφεται η λειτουργικότητα και η διεπαφή και δίνεται μια σύντομη εικόνα της αρχιτεκτονικής.
\par
Στο τέταρτο κεφάλαιο γίνεται η τεχνική ανάλυση της υλοποίησης της εφαρμογής. Παρουσιάζονται διεξοδικά οι πτυχές της υλοποίησης με αναφορές στις προγραμματιστικές λεπτομέρειες.
\par
Στο πέμπτο κεφάλαιο αναλύεται το κομμάτι του server. Πρώτα γίνεται μια εισαγωγή στο Elasticsearch και την ορολογια που το συνοδεύει. Ύστερα γίνεται η τεχνική ανάλυση του Indexer και του client.
\par
Τέλος γίνεται η αποτίμηση της εργασίας και δίνονται προτάσεις βελτίωσης και επέκτασης της εφαρμογής.

\end{section}


\begin{section}{Εργαλεία που χρησιμοποιήθηκαν}
\bigskip
Για την ανάπτυξη της Android εφαρμογής χρησιμοποιήθηκαν τα εξής εργαλεία:
\begin{itemize}
\item Android SDK Platform έκδοση 19.
\item η βιβλιοθήκη Google Play Services για τη χρήση των Google Maps κα Location Services.
\item η βιβλιοθήκη Spring for Android για τα REST calls στον απομακρυσμένο server. \cite{spring_and_proj}
\item η βιβλιοθήκη Jackson για marshalling/unmarshalling των JSON documents. \cite{jackson}
\item η βιβλιοθήκη Google Maps Android API utility library για τη σχεδίαση των σημείων. \cite{map_utils}
\item η ανάπτυξη έγινε στο Android Studio 0.8.9 IDE με το εργαλείο Gradle για dependency management.
\end{itemize}

Στο back-end χρησιμοποιήθηκαν τα εξής:
\begin{itemize}
\item Spring Boot με embedded Tomcat server. \cite{spring_boot_proj}
\item Elasticsearch server (1.3.2) ως NoSQL data-store και search server. \cite{es_home}
\item το Java API του elasticsearch (1.3.2) για indexing και search. \cite{es_java}
\item η ανάπτυξη έγινε στο Netbeans 8.0.1 IDE με το εργαλείο Maven για dependency management.
\end{itemize}

Τα δεδομένα των πρατηρίων μεταφορτώθηκαν από τον κρατικό ιστότοπο \url{data.gov.gr} σε μορφή Microsoft Excel (.xlsx) και αφορούν το μήνα Σεπτέμβριο 2013.

Το παρόν στοιχειοθετήθηκε με \XeLaTeX.

\end{section}
